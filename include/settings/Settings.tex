% CREATED BY DAVID FRISK, 2016

%For HOL declarations
\usepackage{holtexbasic,proof}
\usepackage[T1]{fontenc}
\usepackage{amsmath}
\usepackage{alltt}
\usepackage{holtexbasic}
\usepackage{flushend}
% \theoremstyle{definition}
\newtheorem{definition}{Definition}
\newtheorem{theorem}{Theorem}
\newtheorem{lemma}{Lemma}

% font from ACM
\usepackage[T1]{fontenc}
\usepackage[tt=false, type1=true]{libertine}
\usepackage[varqu]{zi4}
\usepackage{txfonts}

% BASIC SETTINGS
\usepackage{moreverb}								% List settings
\usepackage{textcomp}								% Fonts, symbols etc.
\usepackage{lmodern}								% Latin modern font
\usepackage{helvet}									% Enables font switching
\usepackage[T1]{fontenc}							% Output settings
\usepackage[english]{babel}							% Language settings
\usepackage[utf8]{inputenc}							% Input settings
\usepackage{newunicodechar}
\usepackage{amsmath}								% Mathematical expressions (American mathematical society)
\usepackage{amssymb}								% Mathematical symbols (American mathematical society)
\usepackage{graphicx}								% Figures
\usepackage{subfig}									% Enables subfigures
\numberwithin{equation}{chapter}					% Numbering order for equations
\numberwithin{figure}{chapter}						% Numbering order for figures
\numberwithin{table}{chapter}						% Numbering order for tables
\usepackage{minted}						    		% Enables source code listings
\usepackage{chemfig}								% Chemical structures
\usepackage[top=3cm, bottom=3cm,
			inner=3cm, outer=3cm]{geometry}			% Page margin lengths
\usepackage{eso-pic}								% Create cover page background
\newcommand{\backgroundpic}[3]{
	\put(#1,#2){
	\parbox[b][\paperheight]{\paperwidth}{
	\centering
	\includegraphics[width=\paperwidth,height=\paperheight,keepaspectratio]{#3}}}}
\usepackage{float} 									% Enables object position enforcement using [H]
\usepackage{parskip}								% Enables vertical spaces correctly
\usepackage{datetime} %date formatting tools
\usepackage{fixltx2e}
\usepackage{listings}
\usepackage{color}
\usepackage{ebproof}
\usepackage{stackengine}
\usepackage{alltt}
\usepackage{setspace}
\usepackage{imakeidx}
\usepackage{keyval}% http://ctan.org/pkg/keyval
\usepackage{environ}% http://ctan.org/pkg/environ
\usepackage{lipsum}% http://ctan.org/pkg/lipsum
\usepackage{doi}
\usepackage{microtype}

% CODE SNIPPET SETTINGS
\definecolor{dkgreen}{rgb}{0,0.6,0}
\definecolor{gray}{rgb}{0.5,0.5,0.5}
\definecolor{mauve}{rgb}{0.58,0,0.82}

\lstset{frame=tb,
  language=Haskell,
  aboveskip=3mm,
  belowskip=3mm,
  showstringspaces=false,
  columns=fullflexible,
  basicstyle={\small\ttfamily},
  numbers=none,
  numberstyle=\tiny\color{gray},
  keywordstyle=\color{blue},
  commentstyle=\color{dkgreen},
  stringstyle=\color{mauve},
  breaklines=true,
  breakatwhitespace=true,
  tabsize=3,
	keepspaces=true
}

% TODO-command
\newcommand{\todoo}[1]{\textcolor{red}{TODO: #1}\PackageWarning{TODO:}{#1!}}

% For derivation trees
\newcommand{\nothing}{\varnothing}

\ebproofnewstyle{small}{
	separation = 1em, rule margin = .5ex,
	center = false,
	template = \footnotesize{\inserttext},
	left template = \footnotesize{\inserttext},
	right template = \footnotesize{\inserttext},
	right label template = \footnotesize\inserttext,
	left label template = \footnotesize\inserttext
}

%inlined code command
% \newcommand{\inc}[1]{\[\texttt{\lstinline{#1}}\]}

% New unicode chars
% \newunicodechar{≅}{\cong}
\newcommand{\expeq}{\ensuremath{\cong}}
\newunicodechar{≅}{\ensuremath{\cong}}
\newcommand{\nin}{\ensuremath{\notin}}
\newunicodechar{∉}{\ensuremath{\notin}}
\newcommand{\imp}{\ensuremath{\Rightarrow}}
\newunicodechar{⇒}{\ensuremath{\Rightarrow}}
% \newcommand{\and}{\ensuremath{\wedge}}
\newunicodechar{∧}{\ensuremath{\land}}

% OPTIONAL SETTINGS (DELETE OR COMMENT TO SUPRESS)

% Disable automatic indentation (equal to using \noindent)
\setlength{\parindent}{0em}

\setlength{\mathindent}{0pt}

% HOL settings
\renewcommand{\HOLTokenTurnstile}{\ensuremath{\vdash\rule{-0.19em}{0em}}}
% \renewcommand{\HOLConst}[1]{\textsf{#1}}
\renewcommand{\HOLFieldName}[1]{\textsf{#1}}
\renewcommand{\HOLSymConst}[1]{\HOLConst{#1}}
\renewcommand{\HOLTyOp}[1]{\HOLConst{#1}}
\renewcommand{\HOLinline}[1]{\textsf{\ensuremath{#1}}}
\renewcommand{\HOLKeyword}[1]{\mathsf{#1}}
\renewcommand{\HOLTokenBar}{\ensuremath{\mathtt{|}}}
\renewcommand{\HOLTokenDoublePlus}{\ensuremath{+\rule{-0.2em}{0em}+}}
\renewcommand{\HOLTokenLeftrec}{\ensuremath{\langle\!|\;}}
\renewcommand{\HOLTokenRightrec}{\ensuremath{\;|\!\rangle}}
\renewcommand{\HOLTokenLsl}{\raisebox{.15em}{\ensuremath{}\scriptsize{\textless\rule{-0.1em}{0em}\textless}{}}}
\newcommand{\TODO}[1]{\textbf{[TODO: {#1}]}}

\newcommand{\progterm}[0]{\Downarrow_{\mathsf{prog}}}
\newcommand{\asmterm}[0]{\Downarrow_{\mathsf{asm}}}
\newcommand{\expterm}[0]{\Downarrow_{\mathsf{exp}}}

\newcommand{\progdiv}[0]{\Uparrow_{\mathsf{prog}}}
\newcommand{\asmdiv}[0]{\Uparrow_{\mathsf{asm}}}
\newcommand{\expdiv}[0]{\Uparrow_{\mathsf{exp}}}

\NewEnviron{holthmenv}{%
	\fontfamily{cmr}\selectfont
  \[
  \scalebox{1.0}{\begin{array}[t]{l}
  \BODY
  \end{array}}
  \]}
\NewEnviron{holthmenvr}{%
  \[
  \scalebox{1.0}{\begin{array}[t]{rll}
  \BODY
  \end{array}}
  \]}
\NewEnviron{holthmenvl}{%
	\begin{equation}\begin{aligned}
		\scalebox{1.0}{\begin{array}[t]{l}\BODY\end{array}}
	\end{aligned}\end{equation}}

\hyphenation{comp-cert}

% Caption settings (aligned left with bold name)
\usepackage[labelfont=bf, textfont=normal,
			justification=justified,
			singlelinecheck=false]{caption}


% Activate clickable links in table of contents
\usepackage{hyperref}
\hypersetup{colorlinks, citecolor=black,
   		 	filecolor=black, linkcolor=black,
    		urlcolor=black}


% Define the number of section levels to be included in the t.o.c. and numbered	(3 is default)
\setcounter{tocdepth}{5}
\setcounter{secnumdepth}{5}


% Chapter title settings
\usepackage{titlesec}
\titleformat{\chapter}[display]
  {\Huge\bfseries\filcenter}
  {{\fontsize{50pt}{1em}\vspace{-4.2ex}\selectfont \textnormal{\thechapter}}}{1ex}{}[]


% Header and footer settings (Select TWOSIDE or ONESIDE layout below)
\usepackage{fancyhdr}
\pagestyle{fancy}
\renewcommand{\chaptermark}[1]{\markboth{\thechapter.\space#1}{}}


% Select one-sided (1) or two-sided (2) page numbering
\def\layout{2}	% Choose 1 for one-sided or 2 for two-sided layout
% Conditional expression based on the layout choice
\ifnum\layout=2	% Two-sided
    \fancyhf{}
	\fancyhead[LE,RO]{\nouppercase{ \leftmark}}
	\fancyfoot[LE,RO]{\thepage}
	\fancypagestyle{plain}{			% Redefine the plain page style
	\fancyhf{}
	\renewcommand{\headrulewidth}{0pt}
	\fancyfoot[LE,RO]{\thepage}}
\else			% One-sided
  	\fancyhf{}
	\fancyhead[C]{\nouppercase{ \leftmark}}
	\fancyfoot[C]{\thepage}
\fi


% Enable To-do notes
\usepackage[textsize=tiny]{todonotes}   % Include the option "disable" to hide all notes
\setlength{\marginparwidth}{2.5cm}


% Supress warning from Texmaker about headheight
\setlength{\headheight}{15pt}
